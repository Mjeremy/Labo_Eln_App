% !TEX TS-program = pdflatex
% !TEX encoding = UTF-8 Unicode

% This is a simple template for a LaTeX document using the "article" class.
% See "book", "report", "letter" for other types of document.

\documentclass[11pt,a4paper]{report} % use larger type; default would be 10pt
%\documentclass[a4paper]{report}
\usepackage[utf8]{inputenc} % set input encoding (not needed with XeLaTeX)
\usepackage[francais]{babel}

\author{Gramaglia Alexis}

%%% Examples of Article customizations
% These packages are optional, depending whether you want the features they provide.
% See the LaTeX Companion or other references for full information.

%%% PAGE DIMENSIONS
\usepackage[T1]{fontenc}
\usepackage{geometry} % to change the page dimensions
\geometry{a4paper} % or letterpaper (US) or a5paper or....
%\usepackage{geometry}
 \geometry{
 a4paper,
 total={170mm,257mm},
 left=20mm,
 top=20mm,
 }
 \usepackage{xspace}
 
% \geometry{landscape} % set up the page for landscape%   read geometry.pdf for detailed page layout information
\usepackage{comment} % utiliser le commande comment pour commenter du code .tex
\usepackage{multicol}
\usepackage{wrapfig} % pour avoir une photo a cote du texte
\usepackage{graphicx} % support the \includegraphics command and options
\usepackage[parfill]{parskip} % Activate to begin paragraphs with an empty line rather than an indent
%%% PACKAGES
\usepackage{booktabs} % for much better looking tables
\usepackage{array} % for better arrays (eg matrices) in maths
\usepackage{paralist} % very flexible & customisable lists (eg. enumerate/itemize, etc.)
\usepackage{verbatim} % adds environment for commenting out blocks of text & for better verbatim
\usepackage{subfig} % make it possible to include more than one captioned figure/table in a single float
%\usepackage{mathabx}
%\usepackage{alltt}
\usepackage{listings}
\usepackage{color}
\usepackage{makeidx}
%\usepackage{stmaryrd}
\usepackage{caption}
\usepackage{xcolor}
\usepackage{hyperref}%avoir les liens actif
% These packages are all incorporated in the memoir class to one degree or another...

%%% HEADERS & FOOTERS
\usepackage{fancyhdr} % This should be set AFTER setting up the page geometry
\pagestyle{fancy} % options: empty , plain , fancy
\renewcommand{\headrulewidth}{0pt} % customise the layout...
\lhead{}\chead{}\rhead{}
\lfoot{}\cfoot{\thepage}\rfoot{}

%%%tableau
\usepackage{array,multirow,makecell}
%%%

%%% SECTION TITLE APPEARANCE
\usepackage{sectsty}
%\usepackage[Lenny]{fncychap}%Lenny,Rejne
\allsectionsfont{\sffamily\mdseries\upshape} % (See the fntguide.pdf for font help)

%%% ToC (table of contents) APPEARANCE
\usepackage[nottoc,notlof,notlot]{tocbibind} % Put the bibliography in the ToC
\usepackage[titles,subfigure]{tocloft} % Alter the style of the Table of Contents
\renewcommand{\cftsecfont}{\rmfamily\mdseries\upshape}
\renewcommand{\cftsecpagefont}{\rmfamily\mdseries\upshape} % No bold!

%%% END Article customizations

\usepackage{charter}%la police du document
\renewcommand{\headrulewidth}{0,1pt}
\fancyhead[C]{\leftmark}
\fancyhead[L]{}
\fancyhead[R]{}

%%%%%%%%%%%% MODIFICATION UD PIED DE PAGE %%%%%%%%%
\renewcommand{\footrulewidth}{1pt}
\fancyfoot[R]{\textbf{page\thepage}} 
\fancyfoot[L]{Gramaglia Alexis}
\fancyfoot[C]{}
%\fancyfoot[R]{\leftmark}
% -----------------------------------------------------



%-------------- CHANGER COULEUR SECTION,SUBSECTION ----------
\usepackage{titlesec}
%\titleformat*{\chapter}{\bfseries\color{green!40!black}}
%\titleformat*{\section}{\bfseries\color[rgb]{0.89,.0,.13}}
%\titleformat*{\section}{\bfseries\color[rgb]{0.0,0.5,0.0}}
%\titleformat*{\subsection}{\bfseries\color[rgb]{1.0, 0.49, 0.0}}
%\titleformat*{\subsubsection}{\bfseries\color{purple!40!black}}
%\titleformat*{\subsubsection}{\bfseries\color[rgb]{.21,.46,.53}}
%--------------------------------------------------------------



\setlength{\parindent}{25pt} % pas d'indentation

%------------- Configuration table des matière -------------
\setcounter{secnumdepth}{2}
\setcounter{tocdepth}{1}
%--------------------------------------------------------------
\usepackage{hyperref}
\hypersetup{
pdfpagemode=UseOutlines,     % UseOutlines, UseThumbs, None, FullScreen : agencement au démarrage
pdfstartview=Fit,            % Fit, FitH, FitB, FitBH : vue de la page au départ (pleine largeur...)
pdffitwindow=true,           % bool: Maximiser
pdfpagelayout=TwoColumnsRight,% SinglePage, TwoColumnsRight/Left, OneColumn : affichage des pages
pdftoolbar=true,             % bool: Affichage de la barre d'outils
pdfmenubar=true,             % bool: Affichage de la barre de menus
bookmarksopen=false,         % bool: Dépliement des signets
bookmarksnumbered=true,      % bool: Numérotation des signets
colorlinks=true,             % bool: Liens colorés
pdfauthor={Gramaglia Alexis},         % Auteur	        % Titre
pdfcreator=PDFLaTeX,         % 
pdfproducer=PDFLaTeX,        %
linkcolor=black,              % Couleur des liens
urlcolor=black,               % url
anchorcolor=black,           % du texte
citecolor=black,             % Couleur de citation 
frenchlinks=true,            % bool: Utiliser des petites majuscules pour les liens, plutôt que de la couleur
pdfborder={0 0 0}             % Ne pas encadrer les liens
}


%---------- Pour pouvoir modifier la police du document ----- %
%\usepackage{helvet}
%\usepackage{tgadventor}
%\usepackage[T1]{fontenc}
%\usepackage{pxfonts} mettre la merde
%\renewcommand{\familydefault}{\sfdefault}
%--------------------------------------------------------------

%\usepackage{minted}% pour la coloration syntaxique
%\newminted[java]{java}{mathescape,linenos,frame=lines,breaklines,framesep=2mm}

\usepackage{tikz}

%------
\usetikzlibrary{backgrounds, calc, shadows, shadows.blur}






% Permet de gérer les numérotations dans la table des matières et dans le texte
\setcounter{secnumdepth}{4}

\definecolor{cas1}{RGB}{20, 107, 31}
\definecolor{cas2}{RGB}{122, 51, 193}
\definecolor{cas3}{RGB}{219, 16, 27}
\definecolor{cas4}{RGB}{209, 118, 20}
\definecolor{cas5}{RGB}{255, 0, 220}
\definecolor{cas6}{RGB}{141, 222, 9}
\definecolor{cas7}{RGB}{9, 127, 224}
\definecolor{cas8}{RGB}{38, 22, 249}
\definecolor{cas6bis}{RGB}{118, 0, 57}

\definecolor{blueDWord}{rgb}{0, 0.09,0.58}
\definecolor{blueWord}{rgb}{0.16, 0.34, 0.63}
\definecolor{greenWord}{rgb}{0.54, 0.72, 0.22}
\definecolor{redWord}{rgb}{0.71, 0.15, 0.17}
\definecolor{violetWord}{rgb}{0.39, 0.25, 0.57}
\definecolor{orangeWord}{rgb}{0.91, 0.44, 0.12}
\definecolor{pinkWord}{rgb}{1, 0.25,1}
\definecolor{powderblue}{rgb}{0.69, 0.88, 0.9}
\definecolor{green}{RGB}{0, 153, 0}

\definecolor{mauve}{rgb}{149, 0, 187}


\usepackage{longtable}


% oNE PAGE SINGLE HORIZONTALE
\usepackage{lscape}

\usepackage{wrapfig}

%tableau long
\usepackage{fourier}
\usepackage{longtable}

% figure H
\usepackage{float}

%http://tex.stackexchange.com/questions/41247/add-pdf-bookmark-manually
%\usepackage[bookmarks=true]{hyperref}
\usepackage{bookmark}

% 3 images 
%http://tex.stackexchange.com/questions/64858/how-to-create-subfloat-figures-two-in-first-row-and-one-below
\usepackage{subfig}


% Pour encadrer les formules
%http://tex.stackexchange.com/questions/122945/coloured-shadowed-boxes-around-equations
\usepackage[skins,theorems]{tcolorbox}
\tcbset{highlight math style={enhanced,
  colframe=red,colback=white,arc=0pt,boxrule=1pt}}

  
  
\usepackage[framemethod=tikz]{mdframed}
  
 %pour les equations
 \usepackage{amsmath}
\usepackage{amsfonts}
\usepackage{amssymb}
  
 % barrer le texte
 \usepackage{cancel}
 
 \usepackage[squaren, Gray, cdot]{SIunits}


 
\setlength\parindent{0pt}

\usepackage[skins,theorems]{tcolorbox}
\tcbset{highlight math style={enhanced,
  colframe=red,colback=white,arc=0pt,boxrule=1pt}}
\newmdenv[innerlinewidth=0.5pt, ,linecolor=red,innerleftmargin=6pt,
innerrightmargin=6pt,innertopmargin=6pt,innerbottommargin=6pt]{mybox}







